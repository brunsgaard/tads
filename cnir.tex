\documentclass[10pt, compress]{beamer}

\usetheme{m}

\usepackage{booktabs}
\usepackage[scale=2]{ccicons}
\usepackage{minted}
\usepackage{textcomp}

\usepgfplotslibrary{dateplot}

\usemintedstyle{trac}

\title{Compact Name-Independent Routing with Minimum Stretch}
\subtitle{}
\date{\today}
\author{Jonas Brunsgaard and Henrik Bendt}
\institute{DIKU}

\begin{document}

\maketitle

\begin{frame}[fragile]
  \frametitle{Introduction}
    Given a weighted undirected network with arbitrary node names, we present a compact couting scheme, using a $\tilde{O}(\sqrt{n})$ space routing table at each node and routing along paths of stretch 3.

    It is known that no compact routing using $o(n)$ space per node can route with stretch below 3. Also, it is known that any stretch below 5 requires $\Omega(\sqrt{n})$ space per node.
\end{frame}

\begin{frame}[fragile]
  \frametitle{Setup}
    Consider an $n$-node weighted undirected graph $G=(V,E,\omega)$.

    Each node $v\in V$:
    \begin{itemize}
        \item is given a unique name with $O(log\; n)$ bits.
        \item each outgoing edge is given a unique port name in $\{1,\dots,deg(v)\}$.
    \end{itemize}
\end{frame}

\begin{frame}[fragile]
  \frametitle{Routing Scheme}
  A distributed alg. that, given destination node's name, allows any node to route messages that will eventually arrive at the destination node.

  \begin{block}{Example: Trivial routing on min cost paths}
    On each node, for each of the possible $(n-1)$ destinations, store a port number leading to the next node on a min cost path to the destination.\\
    Requires each node to store $\Omega (n\; log\; n)$ bits.\\
    Does not scale very well.
  \end{block}
\end{frame}

\begin{frame}[fragile]
  \frametitle{Minimizing parameters}
  \begin{description}
    \item[Stretch] The max ratio over all source-destination pairs between the cost of the path taken by the routing scheme and the cost of a min cost path.
    \item[Memory] The max number of bits over all nodes stored for the routing scheme.
  \end{description}
\end{frame}

\begin{frame}[fragile]
  \frametitle{Labeled Routing}

  %TODO HB

  %evt. some results on this scheme?

\end{frame}

\begin{frame}[fragile]
  \frametitle{Name-Independent Routing}

  %TODO HB

  This is the focus for this presentation.
  %evt. some results on this scheme?

\end{frame} 


\section{The Stretch 3 Scheme}

\begin{frame}[fragile]
  \frametitle{Vicinity Balls}

  %TODO HB

\end{frame}

\begin{frame}[fragile]
  \frametitle{Coloring}

  %TODO HB

\end{frame}

\begin{frame}[fragile]
  \frametitle{Hashing Names To Colors}

  %TODO HB

\end{frame}

\section{Stretch 3 for Complete Graph}

\begin{frame}[fragile]
  \frametitle{Storing}

  %TODO HB

\end{frame}


\begin{frame}[fragile]
  \frametitle{Routing}

  %TODO HB

\end{frame}


\section{Stretch 3 Scheme (for all graphs)}

\begin{frame}[fragile]
  \frametitle{Routing on Trees}

  %JB
  %TODO 3.5

\end{frame}

\begin{frame}[fragile]
  \frametitle{Landmarks}

  %HB
  %TODO 3.6

\end{frame}

\begin{frame}[fragile]
  \frametitle{Partial shortest path trees}

  %JB
  %TODO 3.7

\end{frame}

\begin{frame}[fragile]
  \frametitle{Storing}

  %JB
  %TODO 3.8, (1)..(3)

\end{frame}

\begin{frame}[fragile]
  \frametitle{Storing}

  %JB
  %TODO 3.8, (4)(a),(b)

\end{frame}

\begin{frame}[fragile]
  \frametitle{Routing}

  %JB
  %TODO

\end{frame}


\section{Results}

\begin{frame}[fragile]
  \frametitle{Results}

  %HB
  %TODO present section 1.1, i.e. construction time, runing time, space complexity

\end{frame}

%Results
%Compare to other routing solutions (as done in the introduction?)

%Other?
%Examples?


\end{document}
