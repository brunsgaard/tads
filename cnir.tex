\documentclass[10pt, compress]{beamer}

\usetheme{m}

\usepackage{booktabs}
\usepackage[scale=2]{ccicons}
\usepackage{minted}
\usepackage{textcomp}

\usepgfplotslibrary{dateplot}

\usemintedstyle{trac}

\title{Compact Name-Independent Routing with Minimum Stretch}
\subtitle{}
\date{\today}
\author{Jonas Brunsgaard and Henrik Bendt}
\institute{DIKU}

\begin{document}

\maketitle

\begin{frame}[fragile]
  \frametitle{Introduction}
    Given a weighted undirected network with arbitrary node names, we present a compact couting scheme, using a $\tilde{O}(\sqrt{n})$ space routing table at each node and routing along paths of stretch 3.

    It is known that no compact routing using $o(n)$ space per node can route with stretch below 3. Also, it is known that any stretch below 5 requires $\Omega(\sqrt{n})$ space per node.
\end{frame}

\begin{frame}[fragile]
  \frametitle{Setup}
    Consider an $n$-node weighted undirected graph $G=(V,E,\omega)$.

    Each node $v\in V$:
    \begin{itemize}
        \item is given a unique name with $O(log\; n)$ bits.
        \item each outgoing edge is given a unique port name in $\{1,\dots,deg(v)\}$.
    \end{itemize}
\end{frame}

\begin{frame}[fragile]
  \frametitle{Routing Scheme}
  A distributed alg. that, given destination node's name, allows any node to route messages that will eventually arrive at the destination node.

  \begin{block}{Example: Trivial routing on min cost paths}
    On each node, for each of the possible $(n-1)$ destinations, store a port number leading to the next node on a min cost path to the destination.\\
    Requires each node to store $\Omega (n\; log\; n)$ bits.\\
    Does not scale very well.
  \end{block}
\end{frame}

\begin{frame}[fragile]
  \frametitle{Minimizing parameters}
  \begin{description}
    \item[Stretch] The max ratio over all source-destination pairs between the cost of the path taken by the routing scheme and the cost of a min cost path.
    \item[Memory] The max number of bits over all nodes stored for the routing scheme.
  \end{description}
\end{frame}

\begin{frame}[fragile]
  \frametitle{Labeled Routing}

  %TODO HB

  %evt. some results on this scheme?

\end{frame}

\begin{frame}[fragile]
  \frametitle{Name-Independent Routing}

  %TODO HB

  This is the focus for this presentation.
  %evt. some results on this scheme?

\end{frame} 


\section{The Stretch 3 Scheme}

\begin{frame}[fragile]
  \frametitle{Vicinity Balls}

  For every integer $k \geq 1$, and for a node $u \in V$, let the \textit{vicinity} of $u$, denoted by $B_k(u)$, be the set consisting of $u$ and the $k$ closest nodes to $u$.

  Set $k=[4 \sqrt{n}log\; n]$ and denote $B(u) = B_k(u)$.\\
  Denote $b(u)$ the radius of $B(u)$, $b(u)=max_{w\in B(u)} d(u,w)$.

  %HB

\end{frame}

\begin{frame}[fragile]
  \frametitle{Coloring}

  Partition nodes into the color sets $C_1,\dots,C_{\sqrt{n}}$, with the following properties:
  \begin{enumerate}
    \item Every color-set has at most $2 \sqrt{n}$ nodes.
    \item Every node has in its vicinity at least one node from every color-set.
  \end{enumerate}

  If node $u\in C_i$, it has ``color $i$''. Denote $c(u)=i$.

  This can be constructed in polynomial-time.

  If every node independently chooses a random color, the properties holds with high probability.

  %HB

\end{frame}

\begin{frame}[fragile]
  \frametitle{Hashing Names To Colors}

  Assuming a mapping $h$ from node names to colors is balanced such that at most $O(\sqrt{n})$ names map to the same color.

  In other words, $h(u)=c(u)$ for each node $u\in V$.

  Each node $u\in V$ should be able to compute $h(w)$ for any destination $w$.

  %HB

\end{frame}

\section{Stretch 3 for Complete Graph}

\begin{frame}[fragile]
  \frametitle{Storing}

  Every node $u$ stores the following:

  \begin{enumerate}
    \item The names of all the nodes in the vicinity $B(u)$ and what port number to use to reach them.
    \item The names of all nodes $v$ such that $c(u)=h(v)$ and what port number to use to reach them.
  \end{enumerate}

  %HB

\end{frame}


\begin{frame}[fragile]
  \frametitle{Routing}

  Routing from $u$ to $v$:
  \begin{enumerate}
    \item If $v\in B(u)$ or $c(u)=h(v)$, then $u$ routes directly to $v$ with stretch 1 (i.e. min cost)
    \item Otherwise, $u$ forwards the packet to $w\in B(u)$ such that $c(w)=h(v)$. Then from $w$ the packet goes directly to $v$.\\
        The stretch is at most $3$ since $d(u,w) + d(w,v) \leq d(u,v)+2d(u,v)$.
  \end{enumerate}

  %HB

\end{frame}


\section{Stretch 3 Scheme (for all graphs)}

\begin{frame}[fragile]
  \frametitle{Routing on Trees}

  %JB
  %TODO 3.5

\end{frame}

\begin{frame}[fragile]
  \frametitle{Landmarks}
  \begin{itemize}
    \item Designate one color to be the landmark color.
    \item Let $L$ denote the set of nodes with this color.
    \item Because of the way coloring works:
    \begin{itemize}
        \item $|L| \leq 2 \sqrt{n}$
        \item For every $v\in V$, $B(v)\cap L \neq \emptyset$
    \end{itemize}
    \item For a node $v\in V$, let $\ell_v$ denote the closest landmark node in $B(v)$.
  \end{itemize}
  %HB
  %3.6

\end{frame}

\begin{frame}[fragile]
  \frametitle{Partial shortest path trees}

  %JB
  %TODO 3.7

\end{frame}

\begin{frame}[fragile]
  \frametitle{Storing}

  %JB
  %TODO 3.8, (1)..(3)

\end{frame}

\begin{frame}[fragile]
  \frametitle{Storing}

  %JB
  %TODO 3.8, (4)(a),(b)

\end{frame}

\begin{frame}[fragile]
  \frametitle{Routing}

  %JB
  %TODO

\end{frame}


\section{Results}

\begin{frame}[fragile]
  \frametitle{Results}

  \begin{description}
    \item[Stretch] 3
    \item[header size] $O(log^2/log\;log\;n)$ bits
    \item[Routing information per node] $\tilde{O}(\sqrt{n})$ bits
    \item[Routing] $O(1)$ time
    \item[Construction time] $\tilde{O}(n|E|)$
  \end{description}

  %HB
  %TODO present section 1.1, i.e. construction time, runing time, space complexity
\end{frame}

\begin{frame}[fragile]
  \frametitle{Results}
  Improves the stretch of Arias et al. [2003] from 5 to 3, the known lower bound.

  Answers the open problem from 1989 [Awerbuch et al. 1990].

  %HB
\end{frame}

%Results
%Compare to other routing solutions (as done in the introduction?)

%Other?
%Examples?


\end{document}
